\section{Conclusion} \label{sec:conc}
Throughout the project a process to read PMT files, and reduce certain components of their noise via multiple methods was established. Following on from this, a versatile and lightweight algorithm for finding signal events was developed and tested rigorously across multiple PMT ROOT files. From these tests, an optimal set of parameters were determined that produced an average efficiency of \textbf{96.3(1.1)}\% across the PMT signal files.

This signal spotting algorithm, labelled \textit{signalspotter}, was then used to determine the dark rates of our individual PMT noise files, with the results being compared to the values provided by Hamamatsu Photonics \cite{hamamatsu}. The results, listed in Table \ref{tab:fulldarkrate} for the dark rates of each PMT file in each testing location were within the range of \textbf{1733} and \textbf{7400} Hz. The calculated dark rates all fall within 3$\sigma$ of the literature values, validating the algorithm's ability to collect signal values across large data files with high efficiency. The uncertainties for the calculated dark rates were considered to be oversized, but this issue was dismissed as the results themselves are sufficient proof to demonstrate the algorithms ability to measure dark rates with decent accuracy, managing to be within 800 Hz of the literature value in all but one case.

Concluding this report, the goal of investigating the data produced by the PMTs in different environments was accomplished, as the data provided was studied extensively. An algorithm was produced that reduced noise within the data, and spotted signal events with high efficiency from large arrays of PMT data. This was then validated further by its use in determining the dark rates of the PMT data files provided, with the results being in general agreement with the literature values \cite{hamamatsu}. The versatility of the algorithm allows for possible use in other settings, such as developing processes for investigating after-pulse and pre-pulse signals. Sadly, these possibilities were out of scope for this report.